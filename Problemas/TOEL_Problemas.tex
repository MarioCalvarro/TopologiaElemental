\documentclass[10pt,a4paper,openright]{book}

%% Formateo del título del documento
\title{Topología elemental \\ Problemas}
\author{Íker Muñoz Martínez}
\date{}

%% Formateo del estilo de escritura y de la pagina
\pagestyle{plain}
\setlength{\parskip}{0.35cm} %edicion de espaciado
\setlength{\parindent}{0cm} %edicion de sangría
\clubpenalty=10000 %líneas viudas NO
\widowpenalty=10000 %líneas viudas NO
\usepackage[top=2.5cm, bottom=2.5cm, left=3cm, right=3cm]{geometry} % para establecer las medidas de los margenes
\usepackage[spanish]{babel} %Para que el idioma por defecto sea español
\usepackage{ulem} % para poder subrayar entornos especiales como las secciones

%% Texto matematico y simbolos especiales
\usepackage{amsmath} %Paquetes para mates
\usepackage{mathtools}
\usepackage{amsfonts} %Paquetes para mates
\usepackage{amssymb} %Paquetes para mates
\usepackage{stmaryrd} % paquete para mates
\usepackage{latexsym} %Paquetes para mates
\usepackage{cancel} %Paquete tachar cosas
\usepackage{accents} %Paquete acentos
\usepackage{stackrel} %Paquete para poner por encima y debajo

%% Ruta de las fotos e inclusion de las mismas
\usepackage{graphicx}
\graphicspath{{./fotos/}}

%% Inclusion de referencias cruzadas por defecto y específicas
\usepackage{hyperref}

%% Paquete para definir y utilizar colores por el documento
\usepackage[dvipsnames,usenames]{xcolor} %activar e incluir colores
    %% definicion de los colores que se van a utilizar en cada cabecera
    \definecolor{capitulos}{RGB}{60,0,0}% gama de colores de los capitulos
    \definecolor{secciones}{RGB}{95,8,5}% gama de colores de las secciones
    \definecolor{subsecciones}{RGB}{140,36,31}% gama de colores de las subsections
    \definecolor{subsubsecciones}{RGB}{188,109,79}% gama de colores de las subsubsections
    \definecolor{teoremas}{RGB}{164,56,32}% gama de colores para los teoremas
    \definecolor{demos}{RGB}{105,105,105} % gama de colores para el cuerpo de las demostraciones

%% Paquete para la edición y el formateo de capítulos, secciones...
\usepackage[explicit]{titlesec}
    %% Definición del estilo de los capítulos, secciones, etc...
    \titleformat{\chapter}[display]{\normalfont\huge\bfseries\color{capitulos}}{}{0pt}{\Huge #1}[\titlerule]
    \titleformat{\section}{\normalfont\Large\bfseries\color{secciones}}{}{0pt}{#1}
    \titleformat{\subsection}{\normalfont\large\bfseries\color{subsecciones}}{}{0pt}{\uline{#1}}
    \titleformat{\subsubsection}{\normalfont\normalsize\bfseries\color{subsubsecciones}}{}{0pt}{#1}

%% Paquete para el formateo de entornos del proyecto
\usepackage{ntheorem}[thmmarks]
    %% Definicion del aspecto de los entornos matematicos del proyecto
    \theoremstyle{break}
    \theoremheaderfont{\normalfont\bfseries\color{teoremas}}
    \theorembodyfont{\itshape}
    \theoremseparator{\vspace{0.2cm}}
    \theorempreskip{\topsep}
    \theorempostskip{\topsep}
    \theoremindent0cm
    \theoremnumbering{arabic}
    \theoremsymbol{}
    \theoremprework{\vspace{0.2cm} \hrule}
    \theorempostwork{\vspace{0.2cm}\hrule}
        \newtheorem*{enun}{Enunciado}

    \theoremprework{\vspace{0.25cm}}
        \newtheorem*{theo}{Teorema}

    \theoremprework{\vspace{0.25cm}}
    	\newtheorem*{coro}{Corolario}

    \theoremprework{\vspace{0.25cm}}
    	\newtheorem*{lema}{Lema}

    \theoremprework{\vspace{0.25cm}}
    	\newtheorem*{prop}{Proposición}

    \theoremheaderfont{\normalfont}
    \theorembodyfont{\normalfont\color{demos}}
    \theoremsymbol{\hfill\square}
    	\newtheorem*{demo}{\underline{Demostración}:}

    \theoremheaderfont{\normalfont}
    \theorembodyfont{\sffamily}
    	\newtheorem*{obs}{\underline{Observación}:}
    	\newtheorem*{ej}{\underline{Solución}:}
    	\newtheorem*{pg}{\underline{Política general}:}
    	\newtheorem*{il}{\underline{Ilustración}:}

%% Definicion de operadores especiales para simplificar la escritura matematica
\DeclareMathOperator{\dom}{dom}
\DeclareMathOperator{\img}{img}
\DeclareMathOperator{\rot}{rot}
\DeclareMathOperator{\divg}{div}
\DeclareMathOperator{\inter}{Int}
\DeclareMathOperator{\adh}{Adh}
\DeclareMathOperator{\fr}{Fr}
\newcommand{\dif}[1]{\ d#1}

%% Paquete e instrucciones para la generacion de los dibujos
\usepackage{pgfplots}
\pgfplotsset{compat=1.17}
\usepackage{tkz-fct}
\usepgfplotslibrary{fillbetween}
\usepackage{tikz,tikz-3dplot}
\tdplotsetmaincoords{80}{45}
\tdplotsetrotatedcoords{-90}{180}{-90}
\usetikzlibrary{arrows}
    %% style for surfaces
    \tikzset{surface/.style={draw=blue!70!black, fill=blue!40!white, fill opacity=.6}}

    %% macros to draw back and front of cones
    %% optional first argument is styling; others are z, radius, side offset (in degrees)
    \newcommand{\coneback}[4][]{
        %% start at the correct point on the circle, draw the arc, then draw to the origin of the diagram, then close the path
        \draw[canvas is xy plane at z=#2, #1] (45-#4:#3) arc (45-#4:225+#4:#3) -- (O) --cycle;
    }
    \newcommand{\conefront}[4][]{
        \draw[canvas is xy plane at z=#2, #1] (45-#4:#3) arc (45-#4:-135+#4:#3) -- (O) --cycle;
    }
    
    \tikzset{middlearrow/.style={decoration={markings, mark= at position 0.5 with {\arrow{#1}},},postaction={decorate}}}
    
    \usetikzlibrary{decorations.markings}
    
    \newcommand{\AxisRotator}[1][rotate=0]{
    \tikz [x=0.25cm,y=0.60cm,line width=.2ex,-stealth,#1] \draw (0,0) arc (-150:150:1 and 1);
    }
    
    \usetikzlibrary{shapes}

\usepackage{enumitem}
\begin{document}
\maketitle
\setcounter{tocdepth}{3}% para que salgan las subsubsecciones en el indice
\tableofcontents
\chapter{Lista 0: Para Empezar}%
\label{cha:lista0}

\section{Número 0.1}
\begin{enun}
Comprobar las leyes distributivas para la unión y la intersección de conjuntos, y las leyes de De Morgan.
\end{enun}
\begin{ej}
\begin{itemize}
\item Ley distributiva de la unión: $(A \cap B) \cup C = (A \cup C) \cap (B \cup C)$

$x \in (A \cap B) \cup C \Leftrightarrow (x \in A \wedge x \in B) \vee x \in C \Leftrightarrow (x \in A \vee x \in C) \wedge (x \in B \vee x \in C) \Leftrightarrow$ $x \in (A \cup C) \cap (B \cup C)$
\item Ley distributiva de la intersección: $(A \cup B) \cap C = (A \cap C) \cup (B \cap C)$

$x \in (A \cup B) \cap C \Leftrightarrow (x \in A \vee x \in B) \wedge x \in C \Leftrightarrow (x \in A \wedge x \in C) \vee (x \in B \wedge x \in C) \Leftrightarrow$ $x \in (A \cap C) \cup (B \cap C)$
\item Leyes de De Morgan: $(A \cup B)^c = A^c \cap B^c \ \& \ (A \cap B)^c = A^c \cup B^c$
\begin{enumerate}[label={(\arabic*)}]
\item $x \in (A \cup B)^c \Leftrightarrow x \notin A \cup B \Leftrightarrow x \notin A \wedge x \notin B \Leftrightarrow x \in A^c \wedge x \in B^c \Leftrightarrow x \in A^c \cap B^c$
\item $A \cap B = A^{cc} \cap B^{cc} \overset{(1)}{=} (A^c \cup B^c)^c$, y tomando complementarios en ambos lados obtenemos la segunda ley.
\end{enumerate}
\end{itemize}
\end{ej}

\section{Número 0.2}
\begin{enun}
Se consideran una aplicación $f: A \rightarrow B$ y subconjutos $A_0 \subset A, B_0 \subset B$.
\begin{enumerate}[label={(\arabic*)}]
 \item Demostrar que $A_0 \subset f^{-1}(f(A_0))$ y que se da la igualdad si $f$  es inyectiva.
 \item Demostrar que $f(f^{-1}(B_0)) \subset B_0$ y que se da la igualdad si $f$ es sobreyectiva.
\end{enumerate}
\end{enun}
\begin{ej}
\begin{enumerate}[label={(\arabic*)}]
 \item Sea $x \in A_0$. Así, $f(x) \in f(A_0)$, y por definición de preimagen, $x \in f^{-1}(f(A_0))$. En particular, obtenemos la igualdad $A_0 
 = f^{-1}(f(A_0))$ si $f$ es inyectiva. Supongamos que $\exists y \in f^{-1}(f(A_0)) \setminus A_0$. Así, $f(y) \in f(A_0)$, luego existe un $x \in A_0$ tal que $f(y) = f(x)$. Como $f$ es inyectiva, $x = y \in A_0$, lo que nos lleva a una contradicción. 
 \item Sea $y \in f(f^{-1}(B_0))$. Así, existe $x \in f^{-1}(B_0)$ tal que $y = f(x)$. Por tanto, si aplicamos f, obtenemos que $f(x) = y \in B_0$. En particular, obtenemos la igualdad $f(f^{-1}(B_0)) = B_0$ si $f$ es sobreyectiva. Si $y \in B_0$, como $f$ es sobreyectiva, existe un $x \in f^{-1} (B_0)$ tal que $f(x) = y$. Aplicamos $f^{-1}$ y obtenemos que $f^{-1}(y) = x \in f^{-1}(B_0)$, y al aplicar $f$ obtenemos $f(x) = y \in f(f^{-1}(B_0))$.
\end{enumerate}
\end{ej}

\section{Número 0.3}
\begin{enun}
Se consideran una aplicación $f: A \rightarrow B$ y colecciones de subconjutos $A_i \subset A, B_i \subset B$.
\begin{enumerate}[label={(\arabic*)}]
 \item Probar que $f^{-1}$ conserva inclusiones, uniones, intersecciones y diferencias
 \begin{enumerate}[label={(\alph*)}]
 \item Si $B_i \subset B_j$, entonces $f^{-1} (B_i) \subset f^{-1}(B_j)$
 \item $f^{-1}  (\bigcup_i B_i) = \bigcup_i f^{-1} (B_i)$
 \item $f^{-1}  (\bigcap_i B_i) = \bigcap_i f^{-1} (B_i)$
 \item $f^{-1} (B_i \setminus B_j) = f^{-1}(B_i) \setminus f^{-1}(B_j)$
 \end{enumerate}
 \item Demostrar que $f$ conserva solamente las uniones y las inclusiones:
 \begin{enumerate}[label={(\alph*)}]
 \item Si $A_i \subset A_j$, entonces $f (A_i) \subset f(A_j)$
 \item $f(\bigcup_i A_i) = \bigcup_i f (A_i)$
 \item $f  (\bigcap_i A_i) \subset \bigcap_i f(A_i)$; se da la igualdad si $f$ es inyectiva.
 \item $f (A_i \setminus A_j) \supset f(A_i) \setminus f(A_j)$; se da la igualdad si $f$ es inyectiva.
 \end{enumerate}
\end{enumerate}
\end{enun}
\begin{ej}
\begin{enumerate}[label={(\arabic*)}]
 \item Probar que $f^{-1}$ conserva inclusiones, uniones, intersecciones y diferencias
 \begin{enumerate}[label={(\alph*)}]
 \item Sea $x \in f^{-1}(B_i)$. Así, $f(x) \in B_i \subset B_j$, y por tanto, $x \in f^{-1}(B_j)$.
 \item Sea $x \in f^{-1}(\bigcup_i B_i) \Leftrightarrow f(x) \in \bigcup_i B_i \Leftrightarrow \exists i \in I : f(x) \in B_i \Leftrightarrow \exists i \in I : x \in f^{-1} (B_i) \Leftrightarrow x \in \bigcup_i f^{-1}(B_i)$.
 \item  Análogamente, sea $x \in f^{-1}(\bigcap_i B_i) \Leftrightarrow f(x) \in \bigcap_i B_i \Leftrightarrow \forall i \in I : f(x) \in B_i \Leftrightarrow \forall i \in I : x \in f^{-1} (B_i) \Leftrightarrow x \in \bigcap_i f^{-1}(B_i)$.
 \item  Sea $x \in f^{-1} (B_i \setminus B_j) \Leftrightarrow f(x) \in B_i \setminus B_j \Leftrightarrow f(x) \in B_i \wedge f(x) \notin B_j \Leftrightarrow x \in  f^{-1}(B_i) \wedge x \notin  f^{-1}(B_j) \Leftrightarrow x \in  f^{-1}(B_i) \setminus  f^{-1}(B_j)$.
 \end{enumerate}
 \item Demostrar que $f$ conserva solamente las uniones y las inclusiones:
 \begin{enumerate}[label={(\alph*)}]
 \item  Sea $x \in f(A_i)$. Entonces, $\exists a \in A_i : f(a) = x \Rightarrow a \in A_i \subset A_j$, luego $x = f(a) \in f(A_j)$.
 \item Sea $x \in f(\bigcup_i A_i) \Leftrightarrow \exists a \in \bigcup_i A_i : f(a) = x \Leftrightarrow \exists i \in I \ \exists a \in A_i : f(a) = x$. Por tanto, $\exists i \in I : f(a) \in f(A_i) \ \& \ f(a) = x  \Leftrightarrow f(a) \in \bigcup_i f(A_i) \ \& \ f(a) = x \Leftrightarrow x \in f\bigcup_i f(A_i)$.
 \item Veamos primero el contenido general, y después el caso en que $f$ es inyectiva.
 \begin{itemize}
 \item Sea $x \in f(\bigcap_i A_i)$. Entonces, $\exists a \in \bigcap_i A_i : f(a) = x \Rightarrow \forall i \in I \ \exists a \in A_i$, luego $\forall i \in I \ \exists f(a) \in f(A_i) \ \& \ f(a) =x$. Por tanto, $\forall i \in I x \in f(A_i)$, es decir, $x \in \bigcap_i f(A_i)$.
 \item (Si $f$ inyectiva) Sea $x \in \bigcap_i f( A_i)$. Entonces, $\forall i \in I : x \in f(A_i)$. Entonces, $\forall i \in I \ \exists a \in A_i : f(a) = x \Rightarrow \exists a \in \bigcap_i A_i : f(a) = x$, y como la $f$ es inyectiva, $x \in f(\bigcap_i A_i)$.
 \end{itemize}
 \item  Veamos primero el contenido general, y después el caso en que $f$ es inyectiva.
 \begin{itemize}
 \item Sea $x \in f(A_i) \setminus f(A_j)$, es decir, $x \in f(A_i) \wedge x \in f(A_j)^c$. Por tanto, $\begin{cases} \exists a \in A_i : f(a) = x \\ \nexists a \in A_j : f(a) = x  \end{cases}\Rightarrow \exists a \in A_i \cap A_j^c : f(a) = x$, es decir, $x \in f(A_i \setminus A_j)$.
 \item (Si $f$ inyectiva) Sea $x \in f(A_i \setminus A_j)$, es decir $\exists a \in A_i \setminus A_j : f(a)=x$. Por tanto, $\begin{cases} \exists a \in A_i : f(a) = x \\ \exists a \notin A_j : f(a) = x  \end{cases}$. Como $f$ es inyectiva, los $a$ son los mismos y por tanto $x \in f(A_i) \cap f(A_j)^c$, es decir, $x \in f(A_i) \setminus f(A_j)$.
 \end{itemize}
 \end{enumerate}
\end{enumerate}
\end{ej}

\section{Número 0.4}
\begin{enun}
Probar que el conjunto $\mathbb{Q}$ de los números racionales es numerable. Probar que el intervalo $[0,1]$ no es numerable, y que por tanto no lo es $\mathbb{R}$.
\end{enun}
\begin{ej}
\begin{enumerate}[label={(\arabic*)}]
\item Veamos primeramente que el conjunto $\mathbb{Q}$ es numerable. Apoyándonos en el hecho de que $\mathbb{Z}$ es numerable, construimos la siguiente aplicación: \begin{align*}
f: \mathbb{Z} \times \mathbb{Z}^* &\rightarrow \mathbb{Q} \\ 
(p,q) &\mapsto \frac{p}{q}
\end{align*}
La aplicación $f$ es sobreyectiva, ya que $\forall x \in \mathbb{Q}, \exists p,q \in \mathbb{Z} : x = \frac{p}{q}, q \neq 0$. Así, $$Card(\mathbb{Q}) \leq Card(\mathbb{Z} \times \mathbb{Z}) = Card(\mathbb{N})$$ Luego $\mathbb{Q}$ es numerable.
\item Para probar que el intervalo $[0,1]$ no es numerable, emplearemos el argumento conocido como \textit{diagonalización de Cantor}. Por reducción al absurdo, supongamos que el intervalo fuere numerable. Si lo fuera, admitiría una posible enumeración $\{r_n\}_n$. Cada uno de los elementos de la sucesión será un $x \in (0,1)$ (no afecta a la numerabilidad que quitemos los extremos, pues son únicamente dos puntos, es decir, un conjunto finito), que podemos expresar utilizando sus cifras decimales $x = 0.\ldots$ La sucesión tendría la pinta:
\begin{align*}
r_1 &= 0.a_1^1a_2^1a_3^1\ldots \\ 
r_2 &= 0.a_1^2a_2^2a_3^2\ldots \\ 
r_3 &= 0.a_1^3a_2^3a_3^3\ldots \\ 
r_4 &= 0.a_1^4a_2^4a_3^4\ldots \\ 
r_5 &= 0.a_1^5a_2^5a_3^5\ldots \\ 
&\vdots
\end{align*}
Cada $a_i^j$ es un número natural comprendido entre $0$ y $9$. Consideramos entonces el siguiente número: $$R= 0.r_1r_2r_3r_4 \ldots: r_i = a_i^i + 1 (\textsf{mod} 10)$$
El número $R$ pertenece al intervalo; no obstante, difiere con todos los $r_i$ en al menos una posición. Es decir, hemos construido un número que no se encuentra en la sucesión $\{r_n\}_n$, por lo que nuestra hipótesis de que el intervalo $[0,1]$ fuera numerable debe ser falsa.

Además, no es difícil comprobar que la aplicación 
\begin{align*}
f: (0,1) &\rightarrow \mathbb{R} \\
x &\mapsto \tan\left[\pi (x - \frac{1}{2})\right]
\end{align*}
Es una biyección. Por tanto, $\mathbb{R}$ no es numerable, y en particular, $Card(\mathbb{R}) = Card([0,1])$
\end{enumerate}

\end{ej}

\section{Número 0.5}
\begin{enun}
(Distancias en $\mathbb{R}^n$) Comprobar que cada una de las siguientes es una distancia en $\mathbb{R}^n$ y estudiar como son las bolas en cada una de ellas.
\begin{align*}
d(x,y) = \sqrt{\sum_i (x_i - y_i)^2} && \rho_1(x,y) = \sum_i | x_i - y_i| && \rho_2 (x,y) = \max_i |x_i -y_i|
\end{align*}
Para la primera, utilizar la \textit{desigualdad triangular} o \textit{de Minkowsky} $$\sqrt{\sum_i (a_i + b_i)^2} \leq \sqrt{\sum_i a_i^2} + \sqrt{\sum_i b_i^2}$$
\end{enun}
\begin{ej}
dgerhgsre
\end{ej}
\chapter{Lista 1: Espacios Topológicos}%
\label{cha:lista1}

\chapter{Lista 2: Aplicaciones continuas}%
\label{cha:lista2}

\chapter{Lista 3: Construcción de topologías}%
\label{cha:lista3}

\chapter{Lista 4: Separación}%
\label{cha:lista4}

\chapter{Lista 5: Numerabilidad}%
\label{cha:lista5}

\chapter{Lista 6: Compacidad}%
\label{cha:lista6}

\chapter{Lista 7: Conexión}%
\label{cha:lista7}

\chapter{Lista 8: Conexión por caminos}%
\label{cha:lista8}

\chapter{Lista 9: Homotopía}%
\label{cha:lista9}

\chapter{Lista 10: Borsuk y sus variantes}%
\label{cha:lista10}
\end{document}
