\chapter{Aplicaciones continuas}%
\label{cha:aplicaciones_continuas}
\section{Continuidad}%
\label{sec:continuidad}
El famoso $\varepsilon-\delta$ en $\mathbb{R}^n \mathcal{T}_u; x_0 \in X,\ f : \overbrace{X}^{\subset \mathbb{R}^p} \rightarrow \overbrace{Y}^{\subset \mathbb{R}^q}$: 
\begin{gather*}        
\forall \varepsilon > 0, \exists \delta > 0: 
\begin{cases}
    \lVert x - x_0 \rVert < \delta \Rightarrow \lVert f\left( x \right) - f\left( x_0 \right) \rVert < \varepsilon \Leftrightarrow\\
    x \in B\left( x_0, \delta \right) \Rightarrow f\left( x \right) \in B\left( f\left( x_0 \right), \varepsilon \right) \Leftrightarrow\\
    f\left( B\left( x_0, \delta \right) \right) \subset B\left( f\left( x_0 \right), \varepsilon \right) 
\end{cases} \Rightarrow\\
\boxed{\forall B\left( f\left( x_0 \right), \varepsilon \right),\ \exists B\left( x_0, \delta \right) \subset f^{-1}\left( B\left( f\left( x_0 \right), \varepsilon \right) \right)} 
.\end{gather*}

\begin{defi}[Continuidad]
Sea $f: X \rightarrow Y$ una función entre espacios topológicos, decimos que es \textbf{continua en $x_0$ $\in X$} si y sólo si:
\[
\forall V^{f\left( x_0 \right)} :  f^{-1}\left( V^{f\left( x_0 \right)} \right) = V^{x_0} 
\]
es decir, la preimagen de cualquier entorno de $f(x_0)$ es entorno de $x_0$.
\end{defi}

\begin{prop}[Composición de continuidades]
Sean $f:X \rightarrow Y$ y $g: Y \rightarrow Z$ funciones continuas en $x_0\in X$ e $y_0\in Y$ tales que $f(x_0) = y_0$, entonces la composición $h = g \circ f$ es una función continua en $x_0$.
\end{prop}
\begin{demo}
Escojamos un entorno $V^{h\left( x_0 \right)}$ de la imagen por $h$ de $x_0$, entonces
\[
h^{-1} V^{h\left( x_0 \right)} = f^{-1}g^{-1}V^{g\left( y_0 \right)} = f^{-1} V^{y_0} = V^{x_0}
\]
\end{demo}

\begin{ej}
\begin{enumerate}
    \item Sea $f: X_{\text{discreta}} \rightarrow Y$, entonces es continua sean cuales sean los conjuntos de partida y de llegada, pues todo es abierto y, en consecuencia, todo es entorno en $\mathcal{T}_{\text{disc}}$.
    \item Sea $f: X \rightarrow Y_{\text{trivial}}$, entonces es continua sean cuales sean los conjuntos de partida y de llegada, pues como $Y$ es el único abierto, entonces es el único entorno $V^{f\left( x \right)}$ para cualquier $f(x)$ y $f^{-1}V^{f\left( x \right)} = f^{-1}Y = X$, que es abierto.
    \item Si una función $f: X \rightarrow Y_{\text{discreta}}$ es continua, entonces $f$ es localmente constante, pues como en la trivial los puntos son abiertos, entonces el punto $\{f\left( x_0 \right)\}$ es entorno $V^{f\left( x_0 \right)}$ de sí mismo. Por tanto, por la continuidad de $f$, $f^{-1}f\left( x_0 \right) = V^{x_0}$, luego $f \equiv f\left( x_0 \right)$ en ese entorno $V^{x_0}$.
    \item Si una función $f: X \rightarrow Y$ es localmente constante, entonces es continua, puesto que si es localmente constante para cualquier $x_0 \in X$ existe un entorno $U^{x_0} : f\mid_{U^{x_0}} \equiv f\left( x_0 \right)$. De este modo, cualquier entorno $V^{f\left( x_0 \right)}$ de la imagen $f(x_0)$ cumple que $f^{-1}V^{f\left( x_0 \right)} \supset U^{x_0}$ y llamando $V^{x_0} = f^{-1} V^{f\left( x_0 \right)}$ entonces vemos que es entorno de $x_0$.
\end{enumerate}
\end{ej}

\begin{prop}[Caracterización de Continuidad]
Sea $f:X\rightarrow Y$ una función entre espacios topológicos, entonces son equivalentes:
\begin{enumerate}
    \item $f$ es continua.
    \item $\forall U \stackrel{ab}{\subset} Y : f^{-1}\left( U \right) \stackrel{ab}{\subset} X$.
    \item $\forall F \stackrel{cerr}{\subset} Y : f^{-1}\left( F \right) \stackrel{cerr}{\subset} X$.
    \item $\forall A \subset Y : f^{-1}\left( \mathring{A} \right) \subset \inter\left( f^{-1}\left( A \right) \right)$.
    \item $\forall A \subset X : f\left( \overline{A} \right) \subset \overline{f\left( A \right)}$.
\end{enumerate}
\end{prop}
\begin{demo}
\begin{enumerate}
    \item $1 \Rightarrow 2)$
    
    Escojamos un abierto cualquiera $W^{\text{ab}} \subset Y$. Por ser abierto, es entorno de todos sus puntos y, en particular, es entorno de las imágenes que puedan caer dentro de dicho abierto, es decir, $\forall x \in f^{-1} (W) : W = V^{f(x)}$. Por continuidad, las preimágenes de entornos de las imágenes son entornos de las preimágenes, luego $\forall x \in f^{-1} W : f^{-1}W = V^x$ y como es entorno de todos sus puntos, entonces $f^{-1} W \stackrel{ab}{\subset} X$.
    \item $2 \Rightarrow 3)$
    
    Como lo que sabemos es que las preimágenes de abiertos son abiertas y los cerrados se definen en términos de abiertos, no nos queda otra estrategia que intentar demostrarlo pasando los cerrados a sus complementarios: los abiertos.
    
    Escojamos un cerrado cualquiera $C \stackrel{cerr}{\subset} Y$ de modo que conocemos que $Y \setminus C \stackrel{ab}{\subset} Y$. Como conocemos el resultado para abiertos, podemos decir que $f^{-1}\left( Y\setminus C \right) \stackrel{ab}{\subset} X$ y conjuntistamente $X \setminus f^{-1}C = f^{-1}\left( Y\setminus C \right)$, luego directamente tenemos que $f^{-1}C \stackrel{cerr}{\subset} X$. 
    \item $3 \Rightarrow 5)$

    \[
        \overline{f\left( A \right)} \stackrel{\text{cerr}}{\subset} Y \Rightarrow^{3)} \underbrace{f^{-1}\overline{f\left( A \right)}}_{\subset f^{-1}f\left( A \right) \supset A} \subset X \Rightarrow \overline{A} \subset f^{-1}\overline{f\left( A \right)} \Rightarrow f\left( \overline{A} \right) \subset \overline{f\left( A \right)} 
    \]
    \item $5 \Rightarrow 4)$
    \begin{gather*}
        Y \setminus \mathring{A}\Rightarrow \overline{Y\setminus A} \supset \overline{f\left( X \setminus f^{-1}A \right)} \stackrel{5)}{\supset} f\left( \overline{X \setminus f^{-1}\left( A \right)} \right) = f\left( X \setminus \inter\left( f^{-1}A \right) \right) \Rightarrow\\
        X \setminus \inter\left( f^{-1}A \right) \subset f^{-1}\left( Y\setminus \mathring{A} \right) = X \setminus f^{-1}\left( \mathring{A} \right)\Rightarrow f^{-1}\left( \mathring{A} \right) \subset \inter\left( f^{-1}A \right) 
    .\end{gather*}

    \item $4 \Rightarrow 1)$
    \begin{gather*}
        V^{f\left( x \right)} \Rightarrow f\left( x \right) \in \inter\left( V^{f\left( x \right)} \right) \Rightarrow x \in f^{-1}\left( \inter\left( V^{f\left( x \right)} \right) \right) \subset \inter\left( f^{-1}V^{f\left( x \right)} \right) \Rightarrow \\
        f^{-1}V^{f\left( x \right) } \text{ entorno de } x.
    \end{gather*}
\end{enumerate}
\end{demo}

\begin{obs}
\begin{enumerate}
    \item Los cuatros primeros enunciados tratan sobre ``imágenes inversas''. Por ejemplo, la segunda dice que $f^{-1}\mathcal{T}_Y \subset \mathcal{T}_X$.
    \item Pensando que un punto adherente es un ``punto límite'', $5$ nos dice que ``la imagen del límite es el límite de la imagen''.
    \item $Id: \left( X, \mathcal{T}_1 \right) \rightarrow \left( X, \mathcal{T}_2 \right)$ es continua $\Rightarrow \mathcal{T}_2 \subset \mathcal{T}_1$. [$Id^{-1}\mathcal{T}_1 = \mathcal{T}_2]$
\end{enumerate}
Y no mencionamos todos los ejemplos conocidos en espacios afines $\mathbb{R}^n$ con $\mathcal{T}_u$.
\end{obs}

\section{Continuidad y subespacios}%
\label{sec:continuidad_y_subespacios}
\begin{prop}
Sea $f: X \rightarrow Y$ una función continua y $Z \subset X$ un subespacio topológico, entonces la restricción $f|_Z : Z \rightarrow Y$ también es continua.
\end{prop}
\begin{demo}
Aplicando la caracterización de continuidad a través de las preimágenes de abiertos, tenemos que:
\[
\forall A \stackrel{ab}{\subset} Y : \left( f|_Z \right)^{-1} \left( A \right) = Z \cap f^{-1} A \stackrel{ab}{\subset} Z
\]
donde este último conjunto es abierto por ser un abierto ambiente cortado con el subespacio (que es como son los abiertos de la topología relativa).
\end{demo}

\begin{prop}[Continuidad por recubrimientos]
Sea $f: X \rightarrow Y$ una función entre espacios topológicos, si se da alguna de las siguientes condiciones: 
\begin{align*}
\begin{cases}
X = \bigcup_{i \in  I} U_i \mbox{ donde } U_i \stackrel{ab}{\subset} X \\
\forall f|_{U_i} : U_i \rightarrow Y \mbox{ cont.}
\end{cases}
& &
\begin{cases}
X = \bigcup_{i =0}^n F_i \mbox{ donde } F_i \stackrel{cerr}{\subset} X \\
\forall f|_{F_i} : F_i \rightarrow Y \mbox{ cont.}
\end{cases}
\end{align*}
entonces la función $f$ del inicio es continua.
\end{prop}
\begin{demo}
	\begin{itemize}
	\item Escojamos un abierto cualquiera $W \stackrel{ab}{\subset} Y$ y veamos si su preimagen $f^{-1}W$ es un abierto en $X$. En primer lugar, por ser $X = \bigcup_{i \in  I} U_i$ podemos escribir que $f^{-1}W = X\cap f^{-1}W = \bigcup_{i \in  I} U_i \cap f^{-1} W = \bigcup_{i \in  I} \left( U_i \cap f^{-1} W\right) = \bigcup_{i \in  I} \left( f|_{U_i} \right)^{-1} W$. Como $\left( f|_{U_i} \right)^{-1}W \stackrel{ab}{\subset} U_i \stackrel{ab}{\subset} X$, entonces $\left( f|_{U_i} \right)^{-1} W \stackrel{ab}{\subset} X$ y de esta manera podemos escribir $f^{-1}W$ como unión de abiertos $\bigcup_{i \in  I} \left( f|_{U_i} \right)^{-1} W$ de $X$, es decir, que $f^{-1}W \stackrel{ab}{\subset} X$.
	
	\item Escojamos un cerradp cualquiera $C \stackrel{cerr}{\subset} Y$ y veamos si su preimagen $f^{-1}C$ es cerrada en $X$. En primer lugar, por ser $X = \bigcup_{i=0}^n F_i$ podemos escribir que $f^{-1}C = X\cap f^{-1}C = \bigcup_{i = 0}^n F_i \cap f^{-1} C = \bigcup_{i = 0}^n \left( F_i \cap f^{-1} C\right) = \bigcup_{i = 0}^n \left( f|_{F_i} \right)^{-1} C$. Como $\left( f|_{F_i} \right)^{-1}C \stackrel{cerr}{\subset} F_i \stackrel{cerr}{\subset} X$, entonces $\left( f|_{U_i} \right)^{-1} C \stackrel{cerr}{\subset} X$ y de esta manera podemos escribir $f^{-1}C$ como unión finita de cerrados $\bigcup_{i = 0}^n \left( f|_{F_i} \right)^{-1} C$ de $X$, es decir, que $f^{-1}C \stackrel{cerr}{\subset} X$.
    \end{itemize}
\end{demo}

\section{Homeomorfismos}%
\label{sec:homeomorfismos}
Recordemos las definiciones de continuidad que hemos visto:
\[
f \text{ continua} \Leftrightarrow f^{-1}\left( \text{abierto} \right) = \text{abierto} \Leftrightarrow f^{-1}\left( \text{cerrado} \right) = \text{cerrado} 
\]
Ahora veamos que ocurre al invertir la relación.
\begin{defi}[Aplicaciones abiertas y cerradas]
Sea $f: X \rightarrow Y$ una función entre espacios topológicos, decimos que es:
\begin{itemize}
\item \textbf{abierta} si y sólo si las imágenes de abiertos son abiertos.
\item \textbf{cerrada} si y sólo si las imágenes de cerrados son cerrados.
\end{itemize}
\end{defi}

\begin{obs}
La caracterización que dimos de la continuidad hacía referencia a las preimagenes de abiertos y cerrados, no a sus imágenes. De hecho, ni la continuidad implica que la aplicación sea abierta o cerrada, ni viceversa.
\end{obs}

\begin{ej}
En la siguiente tabla podemos ver distintos ejemplos de funciones que verifican algunas de las condiciones que hemos definido, pero no otras simultáneamente:
\begin{center}
\begin{tabular}{c|c|c|c}
funcion & continua & abierta & cerrada \\
\hline
$Id: X_{\text{trivial}} \rightarrow X_{\text{discreta}}$ & \ding{55} & \checkmark & \checkmark \\
\hline
$Id: X_{\text{discreta}} \rightarrow X_{\text{trivial}}$ & \checkmark & \ding{55} & \ding{55} \\
\hline
$j: \left[ 0, 1 \right] \subset \mathbb{R}_{u}$ & \checkmark & \ding{55} & \checkmark \\
\hline
$j: \left( 0, 1 \right) \subset \mathbb{R}_u$ & \checkmark & \checkmark & \ding{55} \\
\hline
\end{tabular}
\end{center}
\end{ej}

\begin{prop}[Trivialidades esenciales]
Sea $f:X\rightarrow Y$ una función biyectiva, entonces las siguientes afirmaciones son equivalentes:
\begin{itemize}
    \item $f$ es abierta
    \item $f$ es cerrada
    \item $f^{-1}$ es continua.
\end{itemize}
\end{prop}
\begin{demo}
\begin{enumerate}
    \item $F_{\text{cerr}} \subset X \Rightarrow X\setminus F_{\text{ab}} \subset X \Rightarrow^{ f\text{ ab}} \underbrace{f\left( X\setminus F \right)}_{= Y\setminus f\left( F \right) \text{(biy.)}} \subset_{\text{ab}} X \Rightarrow f\left( F \right) \subset_{\text{cerr}} Y \Rightarrow f$ cerr.

    \item $F_{cerr} \subset X \Rightarrow^{f \text{cerr}} \underbrace{f\left( F \right)}_{= \left( f^{-1} \right)^{-1} \left( F \right) \text{(biy)}} \subset_{\text{cerr}} Y \Rightarrow f^{-1}$ cont.

    \item $U_{\text{ab}} \subset X \Rightarrow^{f^{-1} \text{cont}} \underbrace{\left( f^{-1} \right) ^{-1} \left( U \right) }_{f\left( U \right) \text{(biy.)}} \subset Y \Rightarrow f$ ab.
\end{enumerate}
\end{demo}

\begin{defi}[Homeomorfismo]
Sea $f: X \rightarrow Y$ una aplicación biyectiva, decimos que es un \textbf{homeomorfismo} si y sólo si $f$ y $f^{-1}$ son continuas, o equivalentemente si:
    \[
    \begin{cases}
        f \text{ biy.}\\
        \text{cont.}\\
        \text{ab.} 
    \end{cases} \Leftrightarrow \begin{cases}
        f \text{ biy.}\\
        \text{cont.}\\
        \text{cerr.}
    \end{cases} 
    \]
\end{defi}

\begin{defi}[Homeomorfismo Local]
Sea $f: X \rightarrow Y$ una función entre espacios topológicos, decimos que es un \textbf{homeomorfismo local} en $x_0 \in X$ si y sólo si $f: V^{x_0} \rightarrow V^{f\left( x_0 \right)}$ restringida a algunos entornos de $x_0$ y $f\left( x_0 \right)$ es homeomorfismo.
\end{defi}

\underline{Ejercicio:} Se pueden tomar $V^{x_0}, V^{f\left( x_0 \right)}$ abiertos.

\begin{obs}
Un homeomorfismo local es abierto.
\end{obs}
\begin{demo}
    $U \subset_{\text{ab}} X \Rightarrow f\left( U \right)$ entorno $\forall y_0 = f\left(\overbrace{x_0}^{\in U}\right) \in f\left( U \right)$.
    Como $f$ homeomorfismo local $\Rightarrow f| : V^{x_0} \rightarrow V^{y_0}$ es homeomorfismo $\Rightarrow f\left( \overbrace{U \cap V^{x_0}}^{\ni y_0 = f\left( x_0 \right)} \right) \subset_{\text{ab}} V^{y_0} \Rightarrow f\left( \overbrace{U \cap V^{x_0}}^{\subset f\left( U \right)} \right)$ entorno de $y_0 \Rightarrow f\left( U \right)$ entorno de $y_0$.
\end{demo}

\begin{ej}[¡Importantes!]
\begin{enumerate}
    \item Proyección estéreo? $\mathbb{S}^{m} \setminus \{\text{punto}\} \rightarrow \mathbb{R}^m$ homeomorfismo.
    \item Proyección exponencial $\mathbb{R} \rightarrow \mathbb{S}': \theta \mapsto e^{2\pi i\theta} = \left( \cos 2\pi \theta, \sin 2\pi \theta \right)$, homeomorfismo local.
    \item Proyección antipodal: $\mathbb{S}^m \rightarrow \mathbb{R}P^{m}: x \mapsto \left[ x \right]$ homeomorfismo local.
    \item Lemniscata: $f: \mathbb{R} \rightarrow X \subset \mathbb{R}^2: t \mapsto \left( \frac{t}{1 + t^4}, \frac{t^3}{1 + t_4} \right)$ es biy. cont, pero \underline{no} homeomorfismo local.

    %TODO: Dibujo Lemniscata
    Engañosamente: 
    \[
        \forall t \in \mathbb{R} \exists \left( t - \varepsilon, t + \varepsilon \right) = I_{\varepsilon}: f| : I_{\varepsilon} \rightarrow f\left( I_{\varepsilon} \right) 
    \]
    es homeomorfismo.

    En $t = 0, f\left( I_{\varepsilon} \right)$ \underline{no} es entorno de $f\left( 0 \right) = \left( 0, 0 \right)$.
\end{enumerate} 
\end{ej}

\begin{defi}[Variedad Topológica]
Una \textbf{variedad topológica} de dimensión $m$ es un espacio localmente homeomorfo a $\mathbb{R}^m$, es decir, que cada punto tiene un entorno abierto homeomorfo a una\footnote{Luego, en el fondo, es homeomorfo a cualquier bola y, por ser éstas parte de una base de $\mathbb{R}^m$, es homeomorfo a todo $\mathbb{R}^m$.} bola $B\left( 0, \varepsilon \right) \subset \mathbb{R}^m$.
\end{defi}
\begin{ej}
Esferas, espacios proyectivos, toros...
\end{ej}

