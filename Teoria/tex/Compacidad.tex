\chapter{Compacidad}%
\label{cha:compacidad}
\section{Concepto y mantras}%
\label{sec:concepto_y_mantras_comp}
\begin{defi}[Compacidad]
Un espacio topológico $X$ decimos que es \textbf{compacto} si y sólo si se cumplen las siguientes condiciones equivalentes:
\begin{itemize}
\item De todo recubrimiento abierto se puede extraer un subrecubrimiento finito:
\[
X = \bigcup_{i \in  I} U_i \Rightarrow \exists U_{i_1} \cup \ldots \cup U_{i_r} = X
\]
\item Dada una familia de cerrados, si la intersección finita de algunos de ellos es no vacía, entones la intersección no finita del total es no vacía:
\[
\forall F_{i_1} \cap \ldots \cap F_{i_r} \neq \emptyset \Rightarrow \bigcap_{i \in  I} F_i \neq \emptyset
\]
\end{itemize}
\end{defi}

\begin{ej}
Como estamos más acostumbrados a la primera de las definiciones que se ha dado, vamos a ejemplificar que la 2ª no se cumple si, por ejemplo, los conjuntos no son cerrados. Para ello, podemos considerar la familia de conjuntos $\{\left( 0, \frac{1}{n} \right)\}_{n = 1}^\infty$. De dicha familia, cualquier cantidad finita tiene intersección no vacía, pero cuando hacemos la intersección arbitraria ésta es vacía.
\end{ej}

\begin{defi}[Subespacios compactos]
Sea $X$ un espacio topológico y $K \subset X$ un subconjunto suyo, decimos que $K$ es \textbf{compacto en $X$} si y sólo si
\[
K \subset \bigcup_{i \in  I} U_i \text{ con } U_i \ab X \Rightarrow \exists U_{i_1} \cup \ldots \cup U_{i_r} \supset K
\]
\end{defi}

\begin{obs}
La definición anterior es trivialmente equivalente a dar la definición de compacidad del inicio del capítulo, pero considerando $K$ como espacio topológico con su topología relativa.
\end{obs}

\begin{ej}
\begin{enumerate}
    \item $K \subset \mathbb{R}_u^n$ es compacto $\Leftrightarrow K$ es cerrado y acotado (Heine-Borel). 
    \item $\left[ a, b \right] \subset \mathbb{R}_u$ compacto. ($\Rightarrow$ Heine-Borel por resultados generales).
    \item Si $X$ es compacto con $\mathcal{T}_{\text{discr.}} \Rightarrow$ es finito 
    \begin{demo}
        $X = \bigcup_{x \in X} \{x\}$ es recubrimiento abierto en $\mathcal{T}_{\text{discreta}}$ y como es compacto $\exists \bigcup_{i=1}^n \left\{ x_i \right\} = X$.
    \end{demo}
    \item $x_k \rightarrow x \Rightarrow K = \{x, x_k: k \ge 1\}$ es compacto.
    \begin{demo}
        Sea $\left\{ U_i \right\}_{i \in I}$ tal que, $\bigcup_{i \in I} U_i = K \Rightarrow$
        \[
            \exists U_{i_0}^x \ni x \xRightarrow{\lim} \begin{rcases}
                x_k \in U_{i_0},\ \forall k > k_0\\
                x_k \in U_{i_k},\ \forall k \le k_0
            \end{rcases} \Rightarrow K \subset U_{i_0} \cup U_{i_1} \cup \ldots \cup U_{i_{k_0}} 
        \]
        Por ser $k_0$ finito.
    \end{demo}

    \item $K \subset \mathbb{R}_u^n$ compacto $\Leftrightarrow \forall A^{\infty} \subset K, A' \cap K \neq \emptyset$ (Bolzano-Weierstrass)
\end{enumerate}
\end{ej}

\begin{prop}[Mantra 1]
Sea $X$ un espacio topológico compacto y $K$ un subespacio cerrado suyo, entonces es compacto.
\[
F \cerr X _{compacto} \Rightarrow F \text{ compacto.}
\]
\end{prop}
\begin{demo}
Dado un recubrimiento $K \subset \bigcup_{i \in I} U_i$ podemos escribir el total en términos de unión de abiertos como $X = \left( X \setminus K \right) \cup \bigcup_{i \in I} U_i$. Como $X$ es compacto, de este recubrimiento podemos extraer otro finito, es decir, $\exists \left( X \setminus K \right) \cup U_{i_1} \cup \ldots \cup U_{i_r} = X$. Por ser subespacio $K \subset X = \left( X \setminus K \right) \cup U_{i_1} \cup \ldots \cup U_{i_r}$ y como no puede estar contenido en su complementario, entonces está $K\subset U_{i_1} \cup \ldots \cup U_{i_r}$, es decir, es compacto.
\end{demo}

\begin{prop}[Mantra 2]
Sea $X$ un espacio topológico compacto y $A\subset X$ un subconjunto con infinitos puntos, entonces tiene puntos de acumulación en $X$.
\[
A_\infty \subset X_{compacto} \Rightarrow A' \neq \emptyset
\]
\end{prop}
\begin{demo}
Vimos en su momento que la adherencia puede expresarse como $\overline{A} = A \cup A'$ y , si $A' = \emptyset$, entonces $\overline{A} = A$ y esto indica que $A$ es cerrado. Cerrado en compacto hemos visto que es compacto luego $A$ es compacto.

Pero recordemos también que al adherencia se podía escribir como
\[
\overline{A} = \{\text{puntos aislados}\} \sqcup A'
\]
es decir, que $A$ está compuesto únicamente por puntos aislados. Como hemos visto que $A$ es compacto y sus puntos son aislados, si proponemos como recubrimiento de abiertos los entornos abiertos disjuntos de cada punto (que existen porque son aislados) tiene que existir un subrecubrimiento finito y, como cada punto sólo pertenece a uno de los conjuntos, no se puede prescindir de ninguno, luego el recubrimiento inicial era finito. Esto indica que existe un número finito de puntos aislados, es decir, que $A$ es finito \#.
\end{demo}

\begin{prop}[Mantra 3]
Sea $f:X\rightarrow Y$ una aplicación continua y $K\subset X$ un subespacio compacto, entonces $f(K)$ es compacto en $Y$.
\end{prop}
\begin{demo}
\[
f\left( X \right) \subset \bigcup_{i} U_i \Rightarrow X = \bigcup_{i} f^{-1} U_i \Rightarrow \exists f^{-1}U_{i_1} \cup \ldots \cup f^{-1}U_{i_r} = X \Rightarrow U_{i_1} \cup \ldots \cup U_{i_r} \supset f\left( X \right)
\]
\end{demo}

\begin{ej}
La propiedad de conservación de la compacidad a través de aplicaciones continuas es muy importante porque permite demostrar la compacidad de espacios muy complicados a través de una aplicación continua desde espacios más sencillos. Por ejemplo, el proyectivo real $\mathbb{R} \mathrm{P}^n$ es compacto porque es imagen continua de la esfera $\mathbb{S}^n$ por la aplicación de la proyección antipodal y la esfera es compacta.
\end{ej}

\begin{prop}[Mantra 4]
Sea $X$ un espacio topológico $T_2$ y $K\subset X$ un subespacio compacto, entonces $K$ es cerrado.
\end{prop}
\begin{demo}
Vamos a ver que $X \setminus K$ es abierto probando que es entorno de todos sus puntos, es decir, vamos a probar que $\forall x\in X\setminus K : \exists U^x \cap K = \emptyset$ (que es lo mismo que $U^x \subset X\setminus K$). 

Dado cualquier punto $x\in X\setminus K$, para cualquier $y\in K$ que escojamos se verifica que existen entornos de ambos $V^x_y$ y $W^y$ disjuntos, por ser el espacio $T_2$. De esta manera, podemos escribir $K := \bigcup_{y\in K} W^y$ y, como es compacto, $K := W^{y_1} \cup \cdots \cup W^{y_r}$.

Como buscamos un abierto $U^x$ que no corte con $K$ y tenemos definido este en términos de unos entornos finitos (de los que hemos calculado otros homólogos que no cortan con ellos), lo único que es posible hacer es considerar el abierto $U^x := V^x_{y_1} \cap \cdots \cap V^{x}_{y_r}$ que es disjunto a cualquier $W^{y_i}$, es decir, a $K$.
\end{demo}

\begin{obs}
En un espacio topológico $T_2$, los resultados que se aplican a puntos se suelen poder aplicar a compactos. Es sencillo, utilizando el Mantra 4, demostrar que dos compactos disjuntos en un $T_2$ se separan como puntos.
\end{obs}

\begin{prop}
Sea $f : X \rightarrow Y$ una función continua, $X$ compacto e $Y$ Hausdorff, entonces $f$ es cerrada.
\end{prop}
\begin{demo}
Si escogemos un cerrado $F \cerr X$, entonces por ser $X$ compacto, $F$ es compacto. Como la aplicación es continua $f(F)$ es compacto en $Y$ y, por ser $T_2$, es cerrado en $Y$. Por tanto, hemos demostrado que la aplicación es cerrada.
\end{demo}

\begin{coro}
Sea $f : X \rightarrow Y$ una función continua, $X$ compacto e $Y$ Hausdorff, si añadimos las siguientes hipótesis:
\[
\begin{cases}
    \text{inyectiva}\\
    \text{sobreyectiva}\\
    \text{biyectiva}
\end{cases} \Rightarrow
\begin{cases}
    \text{inmersión cerrada}\\
    \text{identificación cerrada}\\
    \text{homeomorfismo}
\end{cases} 
\]
se tienen los anteriores resultados.
\end{coro}
\begin{demo}
Por las caracterizaciones que hicimos en el tema de \nameref{cha:construcciones}.
\end{demo}

\section{Tabla de comportamiento}%
\label{sec:tabla_de_comportamiento_comp}
En este apartado estudiamos como se comporta la compacidad con respecto a las construcciones del tema \nameref{cha:construcciones} para ver cuándo se conservan, cuándo se pierden y qué podemos añadir para no perderlas.

%TODO: Fix tabla
\begin{table}[H]
\begin{center}
\begin{minipage}{0.9\linewidth}
\centering
\begin{tabular}{| c | c | c | c | c |}
\hline
& Subespacios & Cocientes & Productos & Sumas\\
\hline
    Compacidad & \begin{tabular}{@{}c@{}}\ding{55}\\ cerrados \checkmark \end{tabular} & \checkmark & \checkmark & \checkmark\\
\hline
    Demostración: & Mantra $1$ & Mantra $3$ & Tychonoff & Unión finita\\
\hline
\end{tabular}
\caption{Tabla de comportamiento de la compacidad respecto a las construcciones, con demostración incluida.}
\end{minipage}
\end{center}
\end{table}

\begin{theo}[de Tychonoff]
Si $X$ e $Y$ son dos compactos $\Rightarrow X \times Y$ es compacto.
\end{theo}
\begin{demo}
Sea $X \times Y = \bigcup_{i \in  I} W_i,\ W_i \in \mathcal{T}_X \times \mathcal{T}_Y$.
\begin{enumerate}
    \item $\forall \left( x, y \right) \in X \times Y,\ \exists U_y^x \times V_x^y \subset W_i$, $i$ depende de $\left( x, y \right)$. Por la definición de la topología producto por base de abiertos.
    \item $\forall x \in X,\ Y = \bigcup_{y \in Y} V_x^y \xRightarrow{Y \text{comp.}} Y = V_x^{y_1} \cup \ldots \cup V_x^{y_r}$, los $y_k$ y su número, $r$, dependen de $x$.
    \item $U^x = U_{y_1}^x \cap \ldots \cap U_{y_r}^x,\ U^x \times V_x^{y_k} \subset U_{y_k}^x \times V_x^{y_k} \subset W_{i_k}$, $i_k$ depende de $x$.
    \item $X = \bigcup_{x \in X} U^x \xRightarrow{X \text{comp.}} X = U^{x_1} \cup \ldots \cup U^{x_s}$.
    \item Veamos que:
    \[
        X \times Y = \bigcup_{\substack{1 \le l \le s\\ 1 \le k \le r}} U^{x_l} \times V_{x_l}^{y_k} \subset \bigcup_{\substack{1 \le l \le s\\ 1 \le k \le r}} W_{i_k} \text{, los } i_k \text{ dependen de los } x_l. 
    \]
    Ya que:
    \begin{itemize}
        \item $x \in X \Rightarrow \exists U^{x_l} \ni x$.
        \item $y \in Y = V_{x_l}^{y_{l_1}} \cup \ldots \cup V_{x_l}^{l_r} \Rightarrow \exists V_{x_l}^{y_{l_k}} \ni y$
    \end{itemize}
    Con lo que tenemos un subrecubrimiento finito a partir de un recubrimiento cualquiera.
\end{enumerate}
\end{demo}

\begin{obs}
\begin{enumerate}
    \item $X \times Y$ compacto $\Rightarrow X$ e $Y$ compactos.\begin{demo}
        Aplicamos el Mantra $3$ para las proyecciones.
    \end{demo} 
    \item Heine-Borel: $K \subset \mathbb{R}_u^n$ cerrado y acotado $\Rightarrow$ compacto.
    \begin{demo}
    Si $K$ es acotado, pertenece al producto cartesiano de intervalos acotados y si además es cerrado, entonces pertenece a un producto cartesiano de intervalos cerrados y acotados de la forma
    \[
	K \subset \left[ a_1, b_1 \right] \times \ldots \times \left[ a_n, b_n \right]
    \]
    Como sabemos que el intervalo $[0,1]$ es compacto, por homeomorfismo, los intervalos $[a,b]$ son compactos. Además, por Tychonov, el producto de ellos también lo es, luego tenemos $K$ cerrado contenido en un compacto, es decir, compacto.
    \end{demo}
\end{enumerate}
\end{obs}


%TODO: Fix
\chapter{Compacidad local}%
\label{cha:compacidad_local}

\begin{defi}[Localmente cerrado]
Sea $X$ un espacio topológico, decimos que un subconjunto $Y \subset X$, es \textbf{localmente cerrado} si y sólo si:
\begin{enumerate}
    \item $\forall y \in Y,\ \exists V^y \ent X: Y \cap V^y \cerr V^y$.
    \item $Y \ab \overline{Y}$.
    \item $\exists U \ab X : Y \cerr U$
\end{enumerate}
donde las condiciones anteriores son equivalentes.
\end{defi}

\begin{obs}
En realidad, podemos reescribir las condiciones anteriores de otra forma para ver más claro lo que estamos queriendo decir:
\begin{enumerate}
\item Si se cumple la propiedad, se cumple para una base de entornos.

Para cualquier entorno $U^y \subset V^y$ también se cumple la propiedad, pues $Y\cap U^y \cerr U^y$ si y sólo si $Y\cap U^y = F \cap U^y$ donde $F$ es un cerrado ambiente, es decir, de $V^y$. Tomamos como $F = V^y \cap Y$ y hemos terminado. 

\item Escrito de otra forma, los abiertos de $\overline{Y}$ son abiertos ambiente cortados con $\overline{Y}$, es decir, $Y = \overline{Y} \cap U$ donde $U \ab X$.

\item De nuevo, los cerrados en $U$ son cerrados ambientes cortados con $U$, es decir que podemos reescribir lo anterior como:
\[
\exists U \ab X \mbox{ y } F\cerr X : Y = F \cap U
\]
\end{enumerate}
\end{obs}

\begin{demo}
\begin{itemize}
    \item 1. $\Rightarrow$ 2) ¿$Y = \overline{Y} \cap \left( \bigcup_{y \in Y} U^y \right)$?
    \begin{itemize}
        \item[$\subset)$] Trivial.
        \item[$\supset)$] Tomamos un elemento $x$:
            \begin{align*}
                x \in \overline{Y} \cap U^y &\xRightarrow{?} x \in \adh_{U^y} \left( Y \cap U^y \right) \stackrel{Y \cap U^y \cerr U^y}{=} Y \cap U^y \subset Y\\
                \forall U^x,\ U^x \subset U^y &\xRightarrow{*} \emptyset \neq Y \cap U^x = \left( Y \cap U^y \right) \cap U^x
            .\end{align*}
            Por tanto, gracias a $*$ demostramos $?$.
    \end{itemize}
    \item 2. $\Rightarrow$ 3) 
	
	Con la observación posterior a la definición la demostración es inmediata, pues que se cumpla dos implica que existe un $U \ab X$ tal que $Y = \overline{Y} \cap U$. Luego si tomamos ese $U$ como el abierto de 3. y $\overline{Y}$ como el cerrado $F$ hemos acabado.
	
    \item 3. $\Rightarrow$ 1)
    
    De nuevo, vuelve a ser trivial por la definición, pues basta con tomar como $V^y$ el abierto $U$ de 3.
\end{itemize}
\end{demo}

Esto es un ejemplo de \underline{localización} de una propiedad topológica $\mathcal{P}$ (en este caso es ser cerrado). Se puede entender como:
\begin{align*}
    &\forall x,\ \exists V^x \text{ que cumple } \mathcal{P} \text{ ó } \\
    &\forall x,\ \exists \mathcal{V}^x \text{ base de entornos que cumplen } \mathcal{P} 
.\end{align*}
A veces son equivalentes (como en este caso), a veces no. El concepto adecuado de localización es mediante \underline{bases de entornos}.

%\begin{ej}
%TODO: Dibujo
%\end{ej}

\section{Compacidad local y mantras}%
\label{sec:compacidad_local_y_mantras}
\begin{defi}
$X$ es \textbf{localmente compacto} si $\forall x \in X,\ \exists \mathcal{V}^x$ base de entornos compactos.
\end{defi}

\begin{ej}
\begin{enumerate}
    \item $\mathbb{R}_u^n$ es localmente compactos: $\mathcal{V}^x = \{B\left[ x, \varepsilon \right] : \varepsilon > 0\}$

    \item $T = B\left( 0, 1 \right) \cup \{p\},\ T_u$ no es localmente compacto.
    \begin{demo}
    Veamos un conjunto infinito sin acumulación en $T$:
    \begin{align*}
        \exists V^p \stackrel{\text{comp.}}{\subset} T &\Rightarrow \exists B\left( 0, \varepsilon \right) \cap T \subset V^p \Rightarrow \exists \overbrace{x_k}^{\in V^p} \rightarrow x_0 \in S\left( 0, 1 \right) \setminus T\\
       &\Rightarrow \{x_k : k \ge 1\} \subset V^p \subset T 
    .\end{align*}
    \end{demo}

    \begin{figure}[H]
        \centering
        \incfig[0.5]{usual-no-localmente-compacto}
        \caption{\textit{Visualización de un conjunto que no es localmente compacto en la topología usual de $\mathbb{R}^2$.}}
        \label{fig:usual-no-localmente-compacto}
    \end{figure}

    \item En general NO basta que exista un entorno compacto.

    En $S = T \sqcup \{q\}$ tomamos como entornos del punto añadido $q$ los $W \subset S$ que tienen complementario finito (y $q \in W$). 
    Es decir, para $q$ tenemos la topología de los complementarios finitos mientras que para el resto de puntos tenemos la topología usual.
    En este caso, $S$ es compacto, pero sus puntos no son localmente compactos (por lo menos, $q$)

    Pero este caso es un ejemplo con un espacio \underline{no separado}.
\end{enumerate}
\end{ej}

\begin{prop}
Si $X$ es $T_2$ y $x \in X$ tiene un entorno compacto, entonces tiene una base de entornos compactos.
\end{prop}
\begin{demo}
    Queremos ver que $\forall U^x$ abierto, $\exists K^x \ent U^x$ compactos.

    $\exists V^x (\stackrel{\text{ab.}}{\supset} W^x)$ compacto $\xRightarrow{?} \mathcal{V}^x =$ \{entornos compactos $K^x$\} \underline{base de entornos}.

    $\exists_{\text{ab.}} U_1^x \subset \overline{U_1^x} \subset U^x$:
    \[ 
        V^x\setminus U^x \cerr V_{\text{comp.}}^x \Rightarrow \overbrace{V \setminus U}^{\not\ni x} \text{comp. en } T_2 \Rightarrow \exists \overbrace{U_1^x\ \&\ A}^{\text{ab. disjuntos}} \supset V^x \setminus U^x
    \]
    Y tenemos que: $K^x = \overline{W^x \cap U^x} \Rightarrow$
    \begin{itemize}
        \item $\overline{V^x}_{\text{comp.}} \cap \overline{U_1^x} = V^x_{\text{comp.}} \cap \overline{U_1^x} \subset V^x \cap \overline{X \setminus A} = V^x \cap \overbrace{\left( X \setminus A \right)}^{\text{cerr.}} \subset U^x$.
        \item Intersección de dos entornos es entorno.
        \item $W^x \cap U_1^x \subset V^x \stackrel{\text{comp. en } T_2}{=} \overline{V^x} \subset X \Rightarrow \underbrace{K^x}_{\text{cerr.}} \subset \underbrace{V^x}_{\text{comp.}} \Rightarrow K^x \text{ comp.}$
    \end{itemize}
\end{demo}

Y tenemos dos mantras:
\begin{prop}[Mantra 1]
Localmente cerrado en localmente compacto es localmente compacto. 
\end{prop}
%TODO: Arreglar formato
\begin{demo}
    Sea $Y \subset X$ con $Y$ localmente cerrado y $X$ localmente compacto e $y \in Y$.

    Tenemos:
    \[
    \overline{U_1^x} \cap V^x \subset \overline{X \setminus A} \cap V^x = \overbrace{\left( X \setminus A \right)}^{\text{cerr.}} \cap V^x \subset U^x
    \]

    Y como $Y$ es localmente cerrado, $\exists W^y \cap Y \cerr W^y$ ent. en $X$. Por ser $X$ localmente compacto $\exists K^y$ compacto tal que, $K^y \subset W^y \Rightarrow K^y \cap W^y \cap Y \cerr K^y \Rightarrow$
    \[
    L^y = \underbrace{K^y \cap W^y}_{\text{ent. en } X} \cap Y \cerr K^y \Rightarrow L^y \text{ ent. en } Y \text{ compacto.}  
    \]
\end{demo}

\begin{prop}[Mantra 2]
Localmente compacto en $T_2$ es localmente cerrado.
\end{prop}
\begin{demo}
Sea $Y \subset X$ con $Y$ localmente compacto, $X$ siendo $T_2$ e $y \in Y \Rightarrow$
\[
\underbrace{\exists L^y}_{\text{comp.}} = \underbrace{V^y \cap Y}_{\text{ent. en } Y} \subset \underbrace{V^y}_{\text{ent. en } X} \xRightarrow{T_2} V^y \cap Y = L^y \cerr V^y.
\]
\end{demo}

\begin{coro}
En un $T_2$ localmente compacto, ser localmente cerrado es equivalente a localmente compacto.
\end{coro}

\section{Tabla de comportamiento}%
\label{sec:tabla_de_comportamiento_loc_comp}
\begin{table}[H]
\centering
\begin{tabular}{| c | c | c | c | c |}
\hline
& Subespacios & Cocientes & Productos & Sumas\\
\hline
    Compacidad local & \begin{tabular}{@{}c@{}}\ding{55}\\ Loc. cerrados \checkmark \end{tabular} & \begin{tabular}{@{}c@{}}\ding{55}\\ Abiertos \checkmark \end{tabular} & \checkmark & \checkmark\\
    \hline
    Demostración: & Mantra $1$ & $f\left( \text{ent.} \right) = $ ent & Tychonoff & Loc. suma es como sum's\\
    \hline
\end{tabular}
\caption{\textit{La tabla nos indica como se conserva la compacidad local en las distintas construcciones que hemos visto. Las sumas y los productos son finitos.}}
\end{table}

\begin{ej}
$Y = \mathbb{R} / \mathbb{Z}$ no es localmente compacto.
\begin{enumerate}
    \item $\mathbb{Z} \subset \underbrace{W}_{\text{ab.}} \subset \mathbb{R}: \exists k + \underbrace{\varepsilon_k}_{0 < \varepsilon_k < 1} \in W\ \forall k \ge 1 \Rightarrow A = \{k + \varepsilon_k : k \ge 1\} \subset W$
    \begin{itemize}
        \item Cerrado
        \item Saturado ($n\mathbb{Z} = \emptyset$)
        \item Infinito
        \item Discreto
    \end{itemize}

    \item $\exists K \subset Y$ entorno compacto de $y = \mathbb{Z} \in Y \Rightarrow \exists \underbrace{W^{\text{ab.}}}_{\supset \mathbb{Z}} \subset p^{-1} K \Rightarrow pA \subset K$ infinito sin acumulación.
\end{enumerate}
\end{ej}

\section{Compactificación por un punto}%
\label{sec:compactificacion_por_un_punto}
Este es otro problema importante: \underline{sumergir un espacio} como subespacio abierto denso de un espacio compacto. Con ``sumergir'' nos referimos a una inmersión.
\[
    X \xhookrightarrow[\text{denso}]{\text{ab.}} X^*
\]
donde $X$ es no compacto y $X^*$ es compacto y Hausdorff, es decir, es localmente compacto y, por ende, la imagen de $X$ es localmente compacta.

Intuitivamente se trata de añadir los límites que el espacio no tiene (por no ser compacto).

\begin{ej}
\begin{enumerate}
    \item $\mathbb{R}^n \equiv B^n \setminus \{a\} \subset \mathbb{S}^n$ vía proyección estereográfica desde $a$.
    \item $\mathbb{R}^n \equiv \mathbb{R}P^n \setminus H \subset \mathbb{R}P^n$ vía cartas afines.
\end{enumerate}
\end{ej}

%TODO: No sé como ponerlo
\begin{defi}
Llamamos \textbf{residuo} a la diferencia entre el espacio desde el que surge la inmersión y el de llegada.
\end{defi}

\begin{prop}
$X$ (no compacto) localmente compacto y $T_2$. Entonces:
\begin{enumerate}
    \item $\exists j : X \xhookrightarrow{} X^*$ compacto $T_2,\ j$ inmersión abierta $X^* \setminus j\left( X \right) = \{\omega\}$.
    \item Unicidad: 
    \begin{figure}[H]
        \centering
            \begin{tikzpicture}[node distance=2cm, auto]
            \node(X) {$X$};
            \node(Y) [above right of=X] {$X_1^*$};
            \node(Z) [below right of=X] {$X_2^*$};
            \draw[right hook->](X) to node {$j_1$}(Y);
            \draw[right hook->](X) to node [below] {$j_2$}(Z);
            \draw[->](Y) to node [right] {$h$}(Z);

            \node(w1) [right=0.5 of Y] {$\omega_1$};
            \node(w2) [right=0.5 of Z] {$\omega_2$};
            \draw[|->](w1) to node [right] {homeo.}(w2);
            \end{tikzpicture}
        \caption{\textit{Con esto vemos que la unicidad es por homeomorfismo.}}
    \end{figure}
\end{enumerate}
\end{prop}
\begin{demo}
\begin{enumerate}
    \item $X^* = X \sqcup \{0\},\ \mathcal{T}^* = \mathcal{T} \cup \{X^* \setminus K: K \subset X \text{ comp.}\}$.
    \begin{itemize}
        \item $\mathcal{T}^*$ es topología: fácil por las hipótesis sobre $X$.
        \begin{itemize}
            \item $K_i \stackrel{\text{comp.}}{\subset} X \xRightarrow{T_2} K_i \cerr X \Rightarrow \overbrace{\bigcap_{i} K_i}^{\text{cerr.}} \subset \overbrace{K_{i_0}}^{\text{comp.}} \Rightarrow \bigcap_{i} K_i$ comp.
            \item $U \ab X,\ X \stackrel{\text{comp.}}{\subset} X \Rightarrow U \setminus K =$ ab. $\setminus$ cerr. $ = $ ab.
            \item $U \ab X,\ K \stackrel{\text{comp.}}{\subset} X \Rightarrow U \cup \left( X^* \setminus K \right) = X^* \setminus \left( K \setminus U \right),\ K \setminus U \subset K$ cerrado $\Rightarrow$ compacto.
        \end{itemize}
        \item $X \subset X^*$ inmersión abierta: $\left( X^* \setminus K \right) \cap X = X \setminus K \in \mathcal{T}$ pues $X$ es $T_2$.
        \item $X^*$ es compacto: $X^* = \bigcup_{i} W_i$.
        \[
        \exists W_{i_0} \ni w \Rightarrow W_{i_0} = X^* \setminus \underbrace{K}_{\text{comp.}} \Rightarrow K \subset W_{i_1} \cup \ldots \cup W_{i_r} \Rightarrow X^* = W_{i_0} \cup W_{i_1} \cup \ldots \cup W_{i_r}  
        \]
        \item $X^*$ es $T_2:$. Si tomamos dos puntos de $X$ simplemente utilizamos que es $T_2$. Por tanto, lo interesante es separar $x \in X$ de $\omega$:
        \[
        x \in X \text{ loc. comp.} \Rightarrow \exists K^x \text{ ent. comp.} \Rightarrow X^* \setminus K^x = U^{\omega} \text{ ent. de } \omega
        \]

        \item $X$ es denso: %TODO
        \[
        \overline{X} = X^*,\ \forall \overbrace{W^x}^{= X \setminus K} \cap X = X \setminus K \neq \emptyset
        \]
        porque $X$ no es compacto, pero $K$ sí.
    \end{itemize}

    \item Unicidad:
    \begin{itemize}
        \item $\begin{rcases}
           h_{j_1} = j_2\\
           j_i \text{ inmersiones} 
        \end{rcases} \Rightarrow h|: j_1\left( X \right) \rightarrow j_2 \left( X \right)$ homeomorfismo.

        \item $h$ continua en $\omega_1$ (análogamente $h^{-1}$ continua en $\omega_2$)
        \begin{align*}
            h\left( \omega_1 \right) = \omega_2 \in W \ab X_2^* &\Rightarrow X_2^* \setminus W \cerr X_2^* \Rightarrow X_2^* \setminus W \stackrel{\text{comp.}}{\subset} j_2\left( X \right)\\
            &\Rightarrow K = h^{-1}\left( X_2^* \setminus W \right) \stackrel{\text{comp.}}{\subset} j_1\left( X \right) \subset X_1^*\\
            \left[ X_1^* \text{ es } T_2 \right] &\Rightarrow K \cerr X_1^* \Rightarrow h^{-1}\left( W \right) = X_1^* \setminus K \ab X_1^*
        .\end{align*}
    \end{itemize}
\end{enumerate}
\end{demo}

\begin{defi}
El espacio $X^*$ se denomina \textbf{compactificación por un punto} de $X$. 

También, \textbf{compactificación de Alexandroff}.
\end{defi}
Por ejemplo, $\mathbb{S}^n$ es la compactificación por un punto de $\mathbb{R}^n$ (vía proyección estéreo como dijimos antes).

\begin{obs}[¡Importante!]
\begin{enumerate}
    \item La unicidad justifica? que un espacio $X^*$ compacto $T_2$ es la compactificación de $X^* \setminus \{0\}$ para cualquier $w \in X^*$.

    \item Si dos espacios son homeomorfos, lo son sus compactos.
    \[
        X_1 \xrightarrow[\text{homeo.}]{f} X_2 \xhookrightarrow{j_2} X_2^* \Rightarrow j_1 = j_2 \circ f : X_1 \rightarrow X_2^*
    \]
    que cumple las condiciones.

    \item Si dos espacios no son homeomorfos, pueden serlo sus compactos.
    \begin{center}
        \includegraphics[scale=0.3]{images/obs_comp_pto} 
    \end{center}
    No son homeomorfos porque %TODO: En clase 
    ningún punto de $X_1$ desconecta sus entornos. Por otro lado, el punto del vértice superior de $X_2$ si que desconecta TODOS sus entornos suficientemente pequeños. 
    Es decir, $X_1$ tiene distintas características topológicas de $X_2$.
\end{enumerate} 
\end{obs}

