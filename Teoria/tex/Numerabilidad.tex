\chapter{Numerabilidad}%
\label{cha:numerabilidad}

\section{Axiomas}%
\label{sec:axiomas}
\subsection{I Axioma}%
\label{sub:i_axioma}
\begin{defi}[I Ax.]
$X$ es \textbf{1\textsuperscript{er} axioma} si $\forall x \in X,\ \exists \mathcal{V}^x$ base numerable de entornos.  
\end{defi}

\begin{obs}
\begin{enumerate}
    \item $\mathcal{B}^x = \{U_k = \mathring{V}_k \}_{k \ge 1}$, base numerable de entornos de abiertos.
    \item $\mathcal{W}^x = \{W_k = U_1 \cap \ldots \cap U_k\}_{k \ge 1}$, base numerable de entornos abiertos encajados.
    \begin{demo}
        Sea $\left\{ V_k^x, k \ge 1 \right\}$ base de entornos abiertos. Hacemos que: $W_1^x = V_1^x,\ W_2^x = V_1^x \cap V_2^x, \ldots,\\ W_k^x = \bigcap_{i = 1}^{k} V_i^x$ y tenemos que:
        \[
        W_{k+1}^x \subset W_{k}^x
        \]
        que mantiene el ser base de entornos abiertos.
    \end{demo}
\end{enumerate}
\end{obs}

\begin{ej}
\begin{enumerate}
    \item $\mathbb{R}^n, \mathcal{T}_u$ cumple el I Axioma
    \begin{demo}    
        Sea $x \in \mathbb{R}^n \Rightarrow \mathcal{W}^x = \{B\left( x, \frac{1}{k} \right): k \ge 1\}$ es base de entornos.
    \end{demo}
    \item $\left( X, \mathcal{T}_a \right), \left( X, \mathcal{T}_{\text{discreta}} \right)$ cumplen el I Axioma.
    \item $\left( \mathbb{R}, \mathcal{T}_{\text{CF}} \right)$ no es I Axioma.
    \begin{demo}
    Sabemos que $\left\{ W_k \right\}$ no es base de entornos de $x$ si $\exists U^x \not \supset W_k\ \forall k \Leftrightarrow \forall k \ge 1,\ \exists \underbrace{x_k}_{\neq x} \in W_k \setminus U^x$.

    Veámoslo por reducción al absurdo. Supongamos que $\exists V^x$ numerable $\Rightarrow$

    %TODO: Creo que no es necesario que sean encajados
    $\exists \mathcal{W}^x = \{W_k\}_{k \ge 1}$ abiertos encajados, $W_k = \mathbb{R} \setminus F_k$ (finito) $\Rightarrow \bigcap_{k \in \mathbb{N}} W_k = \mathbb{R} \setminus \bigcup_{k \in \mathbb{N}} F_k \neq \emptyset \Rightarrow \exists y \in \bigcap_{k \in \mathbb{N}} W_k \Rightarrow U^x = \mathbb{R}\setminus \{y\} \not \supset W_k,\ \forall k \in \mathbb{N}$ (porque $y$ pertenece a todos los $W_k$). Por tanto, tenemos un entorno que no contiene a ningún $W_k$.
    \end{demo}

    \item $\left( \mathbb{R}, \mathcal{T}_{\text{rad}} \right)$ no es I axioma.
    \begin{demo}
    %TODO: Dibujo 
    Sea $\left\{ W_k : W_k \ni x_0 \right\}_{k \ge 1}$ y $L_k:$ recta de pendiente $\frac{1}{k}$ por $x_0$. %TODO: Creo?

    Si hacemos $W_k \cap L_k \supset \left( x_0 - \varepsilon, x_0 + \varepsilon \right) \ni x_k$ 
    tal que $0 < \lVert x_k - x_0 \rVert < \frac{1}{k}$. Ya vimos que $U^{x_0} = \mathbb{R}^2 \setminus \left\{ x_k: k \ge 1 \right\}$ es abierto radial que 
    no contiene ningún $W_k$ ya que a estos pertenecen los $x_k$.
    \end{demo}
\end{enumerate}
\end{ej}

\subsection{II Axioma}%
\label{sub:iiax}
\begin{defi}[II Ax.]
$X$ es \textbf{2º axioma} si $\exists \mathcal{B}$, base numerable de abiertos
\end{defi}

\begin{ej}
\begin{itemize}
    \item $\mathcal{T}_a, \mathcal{T}_{\text{discr.}}$ en $X$ no numerable no es II axioma.
    \begin{demo}
        En $\mathcal{T}_a$ la base de abiertos mínima es $\mathcal{B}_a = \left\{ \left\{ a, x \right\}: x \in X \right\}$, por tanto, si $X$ no es numerable
        tampoco lo será esta base.

        Por otro lado, en $\mathcal{T}_{\text{discr.}}$ tenemos como base mínima $\left\{ \left\{ x \right\}: x \in X \right\}$ que no es numerable.
    \end{demo}
    \item $\left( \mathbb{R}^n, \mathcal{T}_{u} \right)$ es II Axioma ya que tenemos como base a: $\mathcal{B} = \{B \left( q, \frac{1}{k}\right) : q \in \mathbb{Q}^n, k \ge 1 \}$.
    \begin{demo}
        Ejercicio.
    \end{demo}
\end{itemize}
\end{ej}

\begin{prop}
\begin{enumerate}
    \item II Ax. $\Rightarrow$ I Ax. 
    \item I Ax $\not \Rightarrow$ II Ax. 
\end{enumerate}
\end{prop}
\begin{demo}
\begin{enumerate}
    \item Sea la base de abiertos: $\mathcal{B} = \{B_k\}_{k \ge 1} \Rightarrow \mathcal{B}^x = \{B_k : x \in B_k\}$ es base de entornos $\forall x \in X$. Si $\mathcal{B}$ es numerable, también lo será $\mathcal{B}^x$.
    \item Tenemos como contraejemplo $\mathcal{T}_{\text{discr.}}$ en $X$ no numerable.
\end{enumerate}
\end{demo}

\subsection{Separable}%
\label{sub:separable}
\begin{defi}[Separable]
$X$ es \textbf{separable} si $\exists A \subset X$, numerable denso.
\end{defi}

\begin{ej}
\begin{itemize}
    \item $\left( \mathbb{R}^n, \mathcal{T}_u \right)$ es separable, porque $\mathbb{Q}^n$ es denso.
    \item $\left( X, \mathcal{T}_{\text{discr.}} \right)$, si $X$ es no numerable, no es separable.
    \item $\mathcal{T}_a$ sí es separable porque $\overline{\left\{ a \right\}} = X$.
    \item $\left( \mathbb{R}, \mathcal{T}_{\text{CF}} \right),\ \forall A \subset \mathbb{R}$, con $A$ numerable infinito, es denso y, por tanto, separable. 
    \item $\left( \mathbb{R}^2, \mathcal{T}_{\text{rad}} \right)$ también lo es. %TODO: Dibujo para ver que Q2 es denso aquí también
\end{itemize}
\end{ej}

\begin{prop}
\begin{enumerate}
    \item II Ax. $\Rightarrow$ separable. 
    \item I Ax. $+$ separable $\not \Rightarrow$ II Ax. 
    \item I Ax. $\not \Rightarrow$ separable. 
    \item Separable $\not \Rightarrow$ I Ax. 
\end{enumerate}
\end{prop}
\begin{demo}
\begin{enumerate}
    \item $\mathcal{B} = \{B_k\}_{k \ge 1} \Rightarrow A = \{a_k : a_k \in B_k\}_{k \ge 1}$ corta a todo abierto (y, por tanto, es denso).
    \item Contraejemplo: $\left( X, \mathcal{T}_a \right), X$ no numerable.
    \item Contraejemplo: Topología discreta en un espacio no numerable.
    \item Contraejemplo: $\left( \mathbb{R}, \mathcal{T}_{\text{CF}} \right) : \overline{\mathbb{Z}} = \mathbb{R}$.
\end{enumerate}
\end{demo}

\subsection{Lindelöf}%
\label{sub:lindelof}
\begin{defi}[Lindelöf]
$X$ es \textbf{Lindelöf} si $\forall X = \bigcup_{i \in I} U_i$ (recubrimiento abierto), $\exists X = \bigcup_{k=1}^{\infty} U_{i_k}$ (subrecubrimiento numerable). 
\end{defi}

Esta forma débil de compacidad se menciona como complemento en los ejercicios.

\section{Tabla de comportamiento}%
\label{sec:tabla_de_comportamiento_num}
%TODO: Fix table
\begin{table}[H]
\centering 
\begin{tabular}{| c | c | c | c | c |}
\hline
& Subespacios & Cocientes & Productos & Sumas\\
\hline
    I Ax. & \checkmark & \begin{tabular}{@{}c@{}}\ding{55}\\ abiertos \checkmark \end{tabular} & \checkmark & \checkmark\\
    \hline
    II Ax. & \checkmark & \begin{tabular}{@{}c@{}}\ding{55}\\ abiertos \checkmark \end{tabular} & \checkmark & \checkmark\\
    \hline
    Separable & \begin{tabular}{@{}c@{}}\ding{55}\\ abiertos \checkmark \end{tabular} & \checkmark & \checkmark & \checkmark\\
    \hline
    Lindelöf & \begin{tabular}{@{}c@{}}\ding{55}\\ cerrados \checkmark \end{tabular} & \checkmark & \ding{55} & \checkmark\\
    \hline
\end{tabular}
\caption{\textit{Tabla que nos indica como se comportan las propiedades que hemos visto en la anterior sección con las distintas construcciones. Las sumas y productos son finitos.}}
\end{table}
%TODO
\begin{demo}
\begin{itemize}
    \item \underline{Subespacios}:
    \begin{itemize}
        \item I Ax. y II Ax. se \underline{heredan} a subespacios intersecando bases.
        \item Separable se hereda a subespacios abiertos intersecando el conjunto denso.

        No en general: Sea $\left( X, \mathcal{T}_a \right)$ con $X$ no numerable $\Rightarrow X \setminus \left\{ a \right\}$ discreto y no Lindelöf.
        \item Lindelöf se hereda a subespacios cerrados como la compacidad. 

        No en general: Sea $Y$ no Lindelöf, $X = Y \cup \{w\}$ compacto, $\mathcal{B}^w = \{X \setminus F: F \subset Y\}$ con $F$ finito.    
    \end{itemize}

    \item \underline{Cocientes}:
    \begin{itemize}
        \item $X = \mathbb{R}_u$ es I y II Axiomas, $Y = \faktor{\mathbb{R}}{\mathbb{Z}}$ no es I.
        \begin{demo}
            $\alpha = \mathbb{Z} \in Y,\ \exists \mathcal{W}^{\alpha} = \{W_k : k \ge 1\}$ abiertos \underline{saturados}, $W_k \supset \mathbb{Z},\ \forall k$

            (Figura \ref{fig:I_ax_II_ax_cocientes}) $\Rightarrow U = \mathbb{R} \setminus \underbrace{\{\varepsilon_k : k \ge 1\}}_{\text{cerr.}}$ entorno abierto saturado de $\mathbb{Z},\ U \not \supset W_k,\ \forall k$. Esto último porque $\varepsilon_k \in W_k$, pero $\varepsilon_k \not\in \mathbb{R} \setminus \left\{ \varepsilon_k : k \ge 1 \right\}$.

        \begin{figure}[H]
            \centering
            \begin{tikzpicture}
                \fill[cyan!15] (-1,0) rectangle (1, -0.14);
                \draw[-] (-2.3,0) -- (2.3,0); 
                \node(O) at (0,0) {$\mid$};
                \node(e) at (-0.7,0) {\textbullet};
                \node(izq) at (-2,0) {$\mid$};
                \node(der) at (2,0) {$\mid$};
                \node at (-1,0) {(};
                \node(Wk) at (1,0) {)};
                \node [below=0.001 of O] {$k$};
                \node [below=0.001 of izq] {$k-1$};
                \node [below=0.001 of der] {$k+1$};
                \node [above=0.001 of e] {$\varepsilon_k$};
                \node [above=0.001 of Wk] {$W_k$};
            \end{tikzpicture}
            \captionsetup{font={color=gray}}
            \caption{\textit{Si $\varepsilon_k \not\in \mathbb{Z}$, vemos que aunque $\mathbb{R}_u$ cumple el I y el II axioma, existe un cociente que no.}}
            \label{fig:I_ax_II_ax_cocientes}
        \end{figure}
        \end{demo}

        \item Las aplicaciones continuas y abiertas conservan I y II.
        \begin{demo}
            La imagen de una base es una base.
        \end{demo}
        \item Las aplicaciones continuas conservan la separabilidad 
        %TODO: Asumo que es suprayectiva, ¿necesario?
        \begin{demo}
            Sabemos que $f\left( \overline{A} \right) \stackrel{\text{cont.}}{\subset} \overline{f\left( A \right)}$. Entonces, si $\overline{A} = X$, $Y = f\left( X \right) = f\left( \overline{A} \right) \subset \overline{f\left( A \right)} \Rightarrow \overline{f\left( A \right)} = Y$.
        \end{demo}
        %TODO: Ya se sabe xd
        \item Las aplicaciones continuas conservan Lindelöf. 
        \begin{demo}
            Como la compacidad, ya se sabe...
        \end{demo}

        Estas tres últimas propiedades se pueden aplicar al cociente porque la proyección es una aplicación continua.
    \end{itemize}

    \item \underline{Productos/sumas} (finitos):
    \begin{itemize}
        \item Para productos: producto \underline{finito} de familias numerables es numerable.
        \item Para sumas: suma \underline{finita} de familias numerables es numerable.
        %TODO: Se asume de nuevo suprayectividad?
        \item La separabilidad se mantiene porque: $\overline{A_1 \times A_2} = \overline{A}_1 \times \overline{A}_2$.
        \item Solo falla Lindelöf:
        \begin{itemize}
            \item $\left( \mathbb{R}, \mathcal{T}_{[, )} \right)$ es Lindelöf (ejercicio no banal).
            %TODO: Dibujo.
            \item $\left( \mathbb{R}^2, \mathcal{T}_{[, )}^2 \right)$ no es Lindelöf: si lo fuera, $L = \{x + y = 0\} \subset \mathbb{R}^2$ heredaría la propiedad, pero es \underline{discreto} no numerable. ($\bot$)
        \end{itemize}
    \end{itemize}
\end{itemize} 
\end{demo}


\section{Sucesiones}
\label{sub:sucesiones}
\begin{defi}[Sucesiones]
Llamamos \textbf{sucesión} en un espacio $X$ a una aplicación $f: \mathbb{N} \rightarrow X$. La denotamos por:
\[
\left\{ x_n \right\}_{n \in \mathbb{N}}
\]
Con esta notación queremos decir que $x_n = f\left( n \right)$.
\end{defi}
\begin{defi}[Límites]
Decimos que $\lim_{k \rightarrow \infty} x_k = x \Leftrightarrow$ 
\[
\forall U^x,\ \exists k_0 \in \mathbb{N}: k \ge k_0 \Rightarrow x_k \in U^x
\]
\end{defi}
\begin{obs}
\begin{enumerate}
    \item $X$ es Hausdorff $\Rightarrow \exists! $ límite. 
    \begin{demo}    
    $x_k \rightarrow x \neq y,\ \exists U^x \cap U^y = \emptyset \Rightarrow \{x_k : k \ge k_0\} \subset U^x$, por tanto, $x_k \not\in U^y \Rightarrow \lim_{n \rightarrow \infty} x_n \neq y$.
    \end{demo}
    En las hojas de ejercicios hay ejemplos en los que el límite no es único.

    \item El I Axioma permite describir la topología con sucesiones:
    \[
     x \in \overline{A} \Leftrightarrow \exists \{x_k\} \subset A: x_k \rightarrow x
    \]

    \begin{demo}
    \begin{itemize}
        \item[$\Rightarrow)$] Supongamos que $x \in \overline{A} \Rightarrow$
        \[
        \begin{rcases}
            \exists \mathcal{W}^x = \{W_k^x\}_{k \ge 1} \text{ base ent. encajados} \xRightarrow{x \in \overline{A}} \exists x_k \in W_k \cap A\\
            \forall U^x \stackrel{\text{base}}{\supset} W_{k_0}^x \stackrel{\text{enc.}}{\supset} W_{k_0 + 1}^x \supset \ldots \Rightarrow x_k \in U^x,\ \forall k \ge k_0  
        \end{rcases} \Rightarrow x_k \rightarrow x
        \]
        \item[$\Leftarrow)$] Supongamos que $\exists \left\{ x_k \right\} \subset A: x_k \rightarrow x$:
        \[
        A \ni x_k \rightarrow x \Rightarrow \forall U^x,\ \exists x_{k_0} \in U^x \cap A
        \]
    \end{itemize}
    \end{demo}
\end{enumerate}
En general, los límites de sucesiones son poco útiles.
\end{obs}

\begin{enun}
Caracterizar la continuidad por sucesiones, si es posible:
\[
f: X \rightarrow Y
\]
\end{enun}

